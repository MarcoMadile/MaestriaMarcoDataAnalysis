%%%%%%%%%%%%%%%%%%%%%%%%%%%%%%%%%%%%%%%%%%%%%%%%%%%%%%%%%%%%%%%%%%%%%%%%%%%%%%%%
% \documentclass[12pt,papel,twoside]{ibtesis}
\documentclass[12pt,screen,twoside,pagebackref]{ibtesis}
% \documentclass[12pt,papel,singlespace,oneside]{ibtesis}
% \documentclass[12pt,papel,preprint,singlespace,oneside]{ibtesis}

\newcommand{\kari}[1]{\textcolor{red}{#1}}

%%%%%%%%%%%%%%%%%%%%% Paquetes extra %%%%%%%%%%%%%%%%%%%%%%%%%%%%%%%%%%%%%%%%%%%
% Por conveniencia: aqu\'{\i} puede cargar todos los paquetes y definir los comandos 
% que necesite
\usepackage{ibextra}
%%%%%%%%%%%%%%%%%%%%%%%%%%%%%%%%%%%%%%%%%%%%%%%%%%%%%%%%%%%%%%%%%%%%%%%%%%%%%%%%
%%%%%%%%%%%%%%%%%%%%% Informacion sobre la tesis %%%%%%%%%%%%%%%%%%%%%%%%%%%%%%%
\title{Estudio del movimiento de poblaciones animales: redes complejas de interacción inspiradas en datos de campo.}
\author{Marco Madile Hjelt}
\director{Dra. Karina F. Laneri }
\codirector{Dr. Luis G. Moyano}
\carrera{Tesis Carrera de Licenciatura en F\'{\i}sica}
\grado{Licenciando}
\laboratorio{ Centro At\'{o}mico Bariloche}
\jurado{Dr. Marcelo N. Kuperman ~ (Instituto Balseiro)} 
%Dr.~Segundo Jurado (Universidad Nacional de Cuyo)\\ 
%Dr.~J.~Otro Jurado (Univ. Nac. de LaCalle)\\
%Dr.~J.~L\'{o}pez Jurado (Univ. Nac. de Mar del Plata)\\
%Dr.~U.~Amigo (Instituto Balseiro, Centro At\'{o}mico Bariloche)
\palabrasclave{Redes complejas, Chelonoidis chilensis, Uso de refugios}
\keywords{Complex neworks, Chelonoidis chilensis, Burrow use}
% Si queremos poner la fecha manualmente:
% \date{Diciembre de 2099}

%%%%%%%%%%%%%%%%%%%%%%%%%%%%%%%%%%%%%%%%%%%%%%%%%%%%%%%%%%%%%%%%%%%%%%%%%%%%%%%%
%\titlepagefalse % Si no quiere compilar la portada descomente esta linea
%\includeonly{apendices} % Compilar s\'{o}lo estos archivos 
\graphicspath{{figs/}} % Lugar donde encontrar las figuras generales (se puede poner uno en cada cap{\'{\i}}tulo)
%%%%%%%%%%%%%%%%%%%%%%%%%%%%%%%%%%%%%%%%%%%%%%%%%%%%%%%%%%%%%%%%%%%%%%%%%%%%%%%%


\begin{document}

% Dentro del environment 'preliminary' va:
% la dedicatoria, resumen, abstract, indices

\begin{preliminary}

% Escriba su dedicatoria
\dedicatoria{
A mi familia\\
A mis amigos
}

%%% \'{I}ndices %%%%

% \begin{abreviaturas}
%                                 %Abreviaturas
% \end{abreviaturas}

\tableofcontents                %\'{I}ndice

\listoffigures                  %Figuras

\listoftables                   %Tablas

\begin{resumen}%
    Sobre la tortuga \textit{Chelonoidis chilensis} se sabe muy poco, es una especie en estado vulnerable  y actualmente, en la zona de estudio, se esta introduciendo ganado; haciendo muy importante el estudio de  sus refugios, su area de movimiento y las relaciones entre tortugas dentro la comunidad. La población de estudio está en la distribución más al sur. Además, por ser reptil se consideran solitarios y se sabe muy poco sobre su red de interacción social.
    En este trabajo, se estudió el movimiento de 27 individuos de \textit{Chelonoidis chilensis} usando dos técnicas de monitoreo: una unidad de navegación  autónoma con GPS y un datalogger comercial. Se implementó un método de filtrado de trayectorias y se construyó una grilla de zonas de interés para las tortugas, utilizando las trayectorias filtradas e interpoladas. Se implementaron dos criterios para identificar los refugios nocturnos de las tortugas. Sobre estos refugios se calculó la distancia media entre refugios al centro de masa (de estos refugios) para machos y hembras y no se encontraron diferencias significativas. Se armaron redes bipartitas de nodos tortugas y refugios y se compararon las proyecciones en nodos tortugas con redes de interacción, armadas a través de los encuentros medidos entre pares de tortugas. Se utilizó la proyección en nodos tortugas de la red bipartita como predictor de enlaces en la red de encuentros y se encontró que las predicciones no son estadísticamente significativas. Se calcularon métricas sobre la topología de la red proyectada y no se encontraron diferencias respecto al uso aleatorio de refugios. Se observaron la existencia de refugios preferidos y se concluyó que la tortuga pasa la noche en refugios cercanos respecto a otros refugios medidos en la zona de medición.
\end{resumen}

\begin{abstract}%
english abstract
\end{abstract}


%%% Local Variables: 
%%% mode: latex
%%% TeX-master: "template"
%%% End: 


\end{preliminary}


% Podemos usar cualquiera de los dos comandos: \input o \include para incluir el texto
\chapter{Introducción}
\chapterquote{Avanza rapido el tema de las tortugas}{D. H. Zanette}

\section{ C. \textit{chilencis} }
\label{S:opciones-que-acepta}

La mayoría de las especies animales son capaces de realizar complejos patrones de movimientos que generalmente dependen del ambiente, factores intrínsecos de los individuos y las interacciones entre ellos (\cite{morales2005adaptive}, \cite{morales2010building} y \cite{nathan2008emerging}). La complejidad de estos movimientos están manifestados en sus trayectorias. Para el caso de las tortugas estas trayectorias dependen fuertemente de la vegetación en la zona de estudio y la época del año (caso que busquen reproducirse, depositar sus huevos, etc.).


Nuestra especie de interés es la tortuga $Chelonoidis$ $chilensis$. Se distribuye desde el Gran Chaco hasta el norte de la Patagonia, como se muestra en la Fig.~\ref{fig:distribuciondeEspecie} (\cite{chebez2008se}). Esta especie está incluida en el Appendix  de la \textit{Convention on International Trade in Endangered Species of Wild Fauna and Flora (CITES)} y fue categorizada como \textit{vulnerable} a nivel nacional \cite{prado2012categorizacion} e internacional por la \textit{International Union for Conservation of Nature (IUCN)}.
Los principales factores que llevaron a esta situación son la reducción, modificación y destrucción de su hábitat, debido a la expansión de la frontera agropecuaria, y su comercialización, siendo la especie nativa de reptiles más ilegalmente traficada en el mercado de mascotas en Argentina (\cite{prado2012categorizacion}). Además, la amenaza a esta especie se ve aumentada con la introducción de especies depredadoras exóticas como el Jabalí (\textit{Sus scrofa}) (\cite{kubisch2014chelonoidis}). En este trabajo estudiaremos una población de tortugas en en el límite sur de su distribución geográfica, a 20 km al norte de San Antonio Oeste, provincia de Río Negro.  \\
    
Las tortugas son animales  herbívoros que se alimentan con tallos y frutos de cactus (\textit{Opuntia sulphurea, Cereus aethiops, Perocactus tuberosus}), gramíneas (\textit{Chloris castilloniana, Trhichloris crinita}), herbáceas (\textit{Alternanthera pugens, Sphaeralcea miniata, S. mendocina, Portulaca grandiflora}) y vainas de leguminosas (\cite{zacarias2016biologia}).
    
    
    
\begin{figure}[h]
    \begin{center}
        \includegraphics[width=\imsize]{figs/Chap1/Chelonoidis_chilensis_geographic_range.png}
        \caption{Distribución geográfica de la especie de tortuga \textit{Chelonoidis chilensis} \label{fig:distribuciondeEspecie}.}
        
        \end{center}
\end{figure}
Esta especie presenta un dimorfismo sexual cuando son adultos. Los machos son notablemente más chicos que las hembras. El período de actividad en la distribución más sur de la especie es el más corto, ya que bruman (parecido a hibernar) por aproximadamente cinco meses. Sus períodos de actividad comienzan en el mes de septiembre y, desde noviembre a diciembre, es cuando el apareamiento es mayormente observado. Entre enero y marzo es cuando las hembras pasan una gran parte del tiempo buscando un lugar adecuado para enterrar sus huevos \cite{Erika}. Todavía falta mucho por aprender acerca de la biología de la población de C. \textit{chilencis} presente en Argentina.
 
Motivados por la falta de información, el objetivo de este estudio es caracterizar el movimiento de las tortugas en una de las poblaciones en el límite sur de su distribución geográfica. Aprender acerca del movimiento de los individuos es fundamental para entender su rol ecológico en el ecosistema y para diseñar políticas de conservación de la especie y su hábitat.


\section{Metodologias}
%%% Local Variables: 
%%% mode: latex
%%% TeX-master: "template"
%%% End: 

\chapter{Redes de interacción entre tortugas}
\graphicspath{{figs/}}

\chapterquote{In retrospect, Euler's unintended message is very simple: Graphs or networks have properties, hidden in their construction, that limit or enhance our ability to do things with them.}{Albert-László Barabási, 1982}

\label{Redes de interacción entre tortugas}
\section{Trayectorias }

Primero se muestran las trayectorias obtenidas para un día de medición (Fig.~\ref{fig:trayeSinFiltr}), como por ejemplo 1/12/2020. Para éstas se realizó un programa en el lenguaje Python utilizando la librería Folium, permitiendo añadir puntos de GPS al mapa \cite{github}.
 
\begin{figure}[ht]
    \begin{center}
       
   
    \includegraphics[width=\imsize]{Chap2/Traye1_12_sinF.png}
\end{center}
    \caption[Trayectorias un dia de medición, sin filtrrar.]{Trayectorias del 1/12/2020, cada color representa una tortuga diferente. Ambas metodologías fueron implementadas, algunos puntos tomados con el tortugómetro escapan a la trayectoria esperada.}
    \label{fig:trayeSinFiltr}
\end{figure}
Se observa en la Fig.~\ref{fig:trayeSinFiltr}, que algunos puntos tomados por el tortugómetro se desvían de la trayectoria esperada para una tortuga (recorren distancias del orden de los kilómetros en menos de 10 minutos). Se estima que estas desviaciones se producen por dos motivos: en primer lugar, en los primeros minutos de medición, el GPS comienza a conectarse a satélites hasta tener la precisión máxima, haciendo que  los primeros puntos tengan una mayor desviación; en segundo lugar, se observó de manera aleatoria la desviación de algún punto respecto de la trayectoria típica.
 
 
 
 
Para corregir estas desviaciones, se implementó un método basado en la velocidad máxima que pueden alcanzar los individuos. El mismo está detallado en el repositorio de GitHub, archivo \textit{CriterioParaSacarData.py} \cite{github}. Para obtener la velocidad máxima, se calculó la distribución de velocidades de la Fig.~\ref{fig:distribuciondeVel}.

 
\begin{figure}[ht]
\begin{center}
       
   
    \includegraphics[width=\imsize]{Chap2/Velocidades2.jpeg}
    \caption[Distribución de velocidades.]{Histograma de velocidades en m/min. Las  velocidades obtenidas mayores a 15 m/min están órdenes de magnitud por encima.}
    \label{fig:distribuciondeVel}
\end{center}
\end{figure}
Se observó en la distribución de velocidades de la Fig.~\ref{fig:distribuciondeVel}, que las tortugas llegan a una velocidad máxima de aproximadamente 15m/min, de manera que se adoptó el criterio de filtrar los tramos de trayectoria en los que la velocidad supera ese valor máximo. Filtrando los puntos de la Fig.~\ref{fig:trayeSinFiltr}, tomando velocidad máxima 15 m/min, se obtuvo  el mapa de la Fig.~\ref{fig:trayeConFiltr}.
 
 
 

\begin{figure}[ht]
    \begin{center}
       
   
    \includegraphics[width=\imsize]{Chap2/Traye1_12_conF.png}
\end{center}
    \caption[Trayectorias un dia de medición, despues del filtrado.]{Trayectorias del 1/12/2020 luego del filtrado, cada color representa una tortuga diferente.}
    \label{fig:trayeConFiltr}
\end{figure}

\section{Zonas de interes}
Partiendo de las trayectorias filtradas, se realizó  una grilla identificando las zonas de recurrencia en la Fig.~\ref{fig:grilla1}. La misma es de gran interés para la investigación y preservación de la especie, en caso que se pudieran identificar los factores o características de las zonas más recurridas, se podrían sugerir políticas de manejo para minimizar los daños sobre las tortugas. Esto es especialmente importante dado que ahora se está introduciendo ganado en la zona con el consiguiente deterioro del hábitat natural de las tortugas. La grilla fue programada en Python y asigna la cantidad de posiciones medidas por el GPS en cada celda \cite{github}.
 
 
\begin{figure}[ht]
    \begin{center}
    \includegraphics[width=\imsize]{Chap2/GrillaSintCNoche.png}
    \end{center}
    \caption[Mapa de zonas de recurrencia.]{Mapa de recurrencias  interactivo con las trayectorias filtradas. Al hacer clic en cualquier celda de la grilla un cartel dice cuantas mediciones fueron tomadas. El tamaño de celda es de 10m$^2$.}
    \label{fig:grilla1}
\end{figure}
 
Se puede observar en la Fig.~\ref{fig:grilla1} un punto que se destaca mucho más que el resto (arriba a la izquierda) teniendo el máximo de mediciones en esa casilla. Esto se debe a que una pareja de tortugas pasó la noche con el tortugómetro puesto en medio de un arbusto de difícil acceso. Para obtener una mejor idea de las zonas de interés diurnas se realizó otra grilla usando sólo datos del tortugómetro registrados en el día (entre 7am y 9pm) y  realizando una interpolación lineal de 1 punto por minuto por cada par de puntos consecutivos (Fig.~\ref{fig:grillaInt}, \cite{github}). Esta interpolación da una aproximación de las casillas por donde tuvo que pasar la tortuga y añade un peso cuando la tortuga se quedó dentro de la misma casilla por una mayor cantidad de tiempo (mediciones consecutivas).
 
\begin{figure}[ht]
    \begin{center}
        
    
    \includegraphics[width=\imsize]{Chap2/GrillaCintSNoche.png}
\end{center}
    \caption[Mapa con zona de recurrencia para trayectorias diurnas.]{Mapa de recurrencias  interactivo con las trayectorias diurnas (7am-9pm) filtradas e interpoladas linealmente. Al hacer clic en cualquier celda de la grilla un cartel dice cuantas mediciones fueron tomadas. El tamaño de celda es de 10m$^2$.}
    \label{fig:grillaInt}
\end{figure}
 
Comparando las Figs. \ref{fig:grilla1} y \ref{fig:grilla1}, se observa en la  Fig.~\ref{fig:grillaInt} un mayor contraste de las otras celdas respecto al que se encuentra arriba a la izquierda. Esto se debe a la extracción de los puntos nocturnos.  Si se puede obtener las características de las zonas de interés, se podrían sugerir políticas de manejo para minimizar los daños sobre las tortugas. Esto es especialmente importante dado que ahora se está introduciendo ganado en la zona con el consiguiente deterioro del hábitat natural de las tortugas.

\section{Red de encuentros}
Partiendo de las trayectorias filtradas, se decidió buscar el solapamiento de las trayectorias, para identificar los encuentros. Para ello, se implementó un codigo en Python, que partiendo de cualquier punto de su trayectoria busca si hay otro punto de otra tortuga que se encuentre a una distancia menor a 20 metros y a una distancia temporal menor a 20 minutos. Cuando se cumple esta condición se van guardando los pares de puntos junto con la hora y el nombre de ambas tortugas.

En las Figs. \ref{fig:encuentros_hora_medida_tortugometro} y \ref{fig:encuentros_hora_medida_igotu}, están la cantidad de encuentros calculados por hora medida por los tortugómetros y por los i-gotU en función de los meses de medición. Los encuentros del tipo macho-hembra fueron normalizados por la cantidad promedio de horas medidas de ambos sexos para cada mes, en cambió para la cantidad de encuentros macho-macho y hembra-hembra se normalizo utilizando la cantidad de horas medidas para cada sexo en cada mes.  
\begin{figure}[ht]
    \begin{center}
       
   
    \includegraphics[width=\imsize]{Chap2/encuentros_por_hora_tortugometro.pdf}
\end{center}
    \caption[Encuentros por hora medida tomando los datos del tortugometro.]{Encuentros sobre cantidad de horas medidas para cada sexo en función de los meses de medición utilizando el tortugometro.Los distintos colores identifican el tipo de encuentro.}
    \label{fig:encuentros_hora_medida_tortugometro}
\end{figure}

\begin{figure}[ht]
    \begin{center}
       
   
    \includegraphics[width=\imsize]{Chap2/encuentros_por_hora_igotu.pdf}
\end{center}
    \caption[Encuentros por hora medida tomando los datos de los i-gotU.]{Encuentros sobre cantidad de horas medidas para cada sexo en función de los meses de medición utilizando los i-gotU. Los distintos colores identifican el tipo de encuentro.}
    \label{fig:encuentros_hora_medida_igotu}
\end{figure}
En la Fig. \ref{fig:encuentros_hora_medida_tortugometro}, se observa que el máximo de encuentros del tipo macho-hembra por hora medida esta en los meses noviembre-diciembre, esto coincide con la epoca de apariamiento. Estos meses estan juntos ya que las mediciones en esos meses fueron tomadas a finales de noviembre y principios de diciembre. Para el mes de enero solo se registraron encuentros del tipo hembra-hembra en ambas figuras ( \ref{fig:encuentros_hora_medida_tortugometro} y \ref{fig:encuentros_hora_medida_igotu}), esto puede deberse a que las hembras están buscando un lugar acorde para depositar sus huevos, haciendo el encuentro hembra-hembra más probable. Para los datos de los i-gotU (\ref{fig:encuentros_hora_medida_igotu}) tambien se registraron encuentros en los meses de febrero, marzo y abril, pero en menor cantidad que en los meses anteriores, asociamos esta diferencia a la disminución de actividad en las tortugas.


Utilizando los encuentros calculados, se armaron dos redes de interacción  en la librería NetworkX \cite{networkx}, una para los datos obtenidos utilizando  tortugometro y otra para los datos provenientes de i-gotU (Figs. \ref{fig:redInteraccion20mincampanas} y \ref{fig:redInteraccion20igotu}). Las conexiones entre nodos tortugas tienen peso linealmente dependiente de la cantidad de encuentros entre ellas, esto se observa en el grosor del link entre dos tortugas y las distancias relativas entre nodos.


\begin{figure}[ht]
    \begin{center}
       
   
    \includegraphics[width=\imsize]{Chap2/red_interaccion_20min_campanas.pdf}
\end{center}
    \caption[Red de encuentros entre tortugas  con datos tomados por el tortugometro.]{Red de encuentros entre tortugas para datos provenientes de las metodología  tortugometro. La condición de encuentro esta dada por una distancia espacial menor a 20 metros y a una distancia temporal menor a 20 minutos.}
    \label{fig:redInteraccion20mincampanas}
\end{figure}



\begin{figure}[ht]
    \begin{center}
       
   
    \includegraphics[width=\imsize]{Chap2/red_interaccion_20min_IGOTO.pdf}
\end{center}
    \caption[Red de encuentros entre tortugas utilizando i-gotU.]{Red de encuentros entre tortugas para datos provenientes de las metodologías i-gotU. La condición de encuentro esta dada por una distancia espacial menor a 20 metros y a una distancia temporal menor a 20 minutos.}
    \label{fig:redInteraccion20igotu}
\end{figure}


\begin{Huge}
Idea : conectar con criterio de encuentros mostrar grafico de encuentros segun la epoca del año y pasar a las dos redes de interacción que tenemos 
\end{Huge}

%%% Local Variables: 
%%% mode: latex
%%% TeX-master: "template"
%%% End: 

\chapter{Uso de refugios}
Para varias especies los refugios son fundamentales para la protección ante predadores y  condiciones climáticas adversas, especialmente para animales de ectotérmicos, como las tortugas que dependen de fuentes externas para la obtención de calor. En especies relativamente solitarias, es esperable que los individuos pasen un tiempo considerable solos en los refugios o tengan pocos encuentros directos fuera de la época de apareamiento \cite{bipartitasTortusPaper}. Ejemplos de estas especies incluyen a los mapaches, zorros rojos, orangutanes y algunas especies de abejas, avispas y murciélagos. Para estas poblaciones de animales salvajes, monitorear y entender estos refugios puede ayudar a establecer patrones sociales de los individuos.
 
En distintos campos cercanos a la zona de medición (San Antonio Oeste, provincia de Río Negro) están introduciendo ganado en el habitat de las tortugas, por lo tanto es importante entender si esta perturbación presenta una amenaza para la integridad de los refugios. Entender el patrón de movimiento  de las tortugas, junto con las caracteristicas geograficas de los refugios mas usados, es de fundamental importancia para el establecimiento de políticas de conservación.
 
\section{Refugios en el mapa}
Para determinar el refugio donde pasó la noche la tortuga se tomaron dos criterios, uno para cada una de las metodologías de medición. Para la metodología de tortugometro, se tomó el último punto registrado de la tortuga en un día de medición y se pidió la condición de que haya sido tomado después de las 20 horas. En base a las anotaciones de campo tomadas por el grupo, las tortugas generalmente  se encontraban en el refugio cuando quitaban los tortugometros después de este horario. A este punto nuevo se le asigna una etiqueta de refugio y un enlace con la tortuga que pasó la noche en ese refugio. A medida que se añade otro refugio primero se verifica que presente una distancia mayor a 20 metros con todos los otros refugios etiquetados. En caso que la distancia sea menor a 20 metros a, por ejemplo el refugio 1, se dice que la tortuga estuvo en el refugio 1.
 
Para los datos tomados por los i-gotU, se decidió mirar primero las distancias recorridas entre el último punto medido (21 horas) de algún día monitoreado y el primer punto del día siguiente (6 horas). También se miraron las distancias entre el primer punto de algún día de monitoreo con el segundo punto (6 horas y 6:15 respectivamente). En la Fig. \ref{fig:distancias} se muestran los histogramas de las distancias para ambos casos. Se observa que entre el último punto de la noche y el primero de la mañana distancias mayores a 20 metros son muy probables, con varias mediciones de distancias del orden de los 50 o 100 metros, lo que nos diría que la tortuga todavía no se encuentra en el refugio a las 21 horas. En cambio, entre el primer punto del día y el segundo punto, las distancias menores a 20 metros son las más probables, lo que nos diría que la tortuga todavía no abandonó el refugio entre las 6 y las 6:15, a excepción de algunos pocos casos donde tenemos distancias del orden de los 50 m. Por ese motivo, para los datos de los i-gotU se decidió tomar el primer punto del día como la posición del refugio y se siguió el mismo procedimiento que con los tortugometos para agregar etiquetas a los refugios. En función a este resultado, para próximas campañas se decidió mantener las mediciones del i-gotU entre las 21 y las 6 horas, pero disminuyendo la frecuencia de muestreo.
 
 
\begin{figure}[ht]
    \begin{center}
        \includegraphics[width=1.4\imsize]{Chap3/Distancias_primer_ult_con_deteterminar_criterio_ref.pdf}
        \caption[Histogramas de las distancias entre los puntos de los datos de i-gotU.]{ Histogramas de las distancias entre los puntos de los datos de i-gotU. Arriba se muestra la distancia entre el último punto del día anterior y el primer punto del día siguiente. Abajo se muestra la distancia entre el primer punto del día y el segundo punto del día.}
        \label{fig:distancias}
        \end{center}
\end{figure}
 
Se graficaron los refugios encontrados en un mapa utilizando la librería Folium, los mapas fueron guardados en formato html para el fácil acceso a los mismos, al clickear en un refugio sobre el html aparece un cartel con las tortugas que pasaron la noche en el refugio. En las Fig. \ref{fig:refus_campanas_con_labels} y \ref{fig:refus_igotu_labels} se muestran los mapas de los refugios encontrados para los datos de las campañas y los datos de i-gotU respectivamente.
 
 
\begin{figure}[ht]
    \begin{center}
        \includegraphics[width=\imsize]{Chap3/map_refugies_with_labels.jpg}
        \caption{Distribución geográfica de los refugios encontrados para los datos provenientes de las campañas.}
        \label{fig:refus_campanas_con_labels}
       
        \end{center}
\end{figure}
 
\begin{figure}[ht]
    \begin{center}
        \includegraphics[width=\imsize]{Chap3/map_refugies_with_labels_IGOTo.jpg}
        \caption{Distribución geográfica de los refugios encontrados para los datos provenientes de los datos de i-gotU.}
        \label{fig:refus_igotu_labels}
       
        \end{center}
\end{figure}
 
En base a observaciones de directas de campo \cite{Erika}, se observaron ciertos machos en distintas zonas de medición, en particular para la epoca de apariamiento. Estas observaciones aisladas nos llevaron a pensar que es posible que  tortugas macho tengan una distribución de refugios más amplia en el espacio que tortugas hembra.  Para poner a prueba esta hipótesis se definieron dos métricas: centro de masa de refugios y distancia media entre refugios. El centro de masa se define como:
\begin{center}
   
 
$$X_{centro}= \sum^{N -1}_{n=0} \frac{i_{n} X_n}{I_{totales}}.$$
\end{center}
Donde $I_{totales}$ es la cantidad de noches donde se registró que la tortuga durmió en un refugio (depende de cada tortuga), $X_n$ es la coordenada X del refugio n, $i_{n}$ es la cantidad de noches que la tortuga durmió en el refugio n y N la cantidad de refugios totales.  Esta cantidad se calcula para todas las tortugas.
Partiendo de $X_{centro}$, la distancia media  espacial de los refugios se calcula como:
$$D = \sum^{N -1}_{n=0} \frac{|X_n i_n - X_{centro}|}{I_{totales}}.$$
\label{eq:distancia_media_refugios}
El valor de $D$ da yba udea de ka extensión en el espacio de los refugios de una dada tortuga. Cuanto mayor es el valor de D, mas dispersos estarán los refugios en el espacio. Esta métrica fue calculada para todas las tortugas y  promediada para  machos y hembras. Se encontró para los machos $\overline{D}_m =  (128\pm66)\,\text{m}$ y para las hembras     $\overline{D}_h = (122\pm82)\,\text{m}$. Es decir que no se encontraron diferencias significativas en la distribución espacial de refugios  entre machos y hembras.
\section{Refugios más usados y caminos tomados}
Dentro de los refugios que utiliza una tortuga, se encontraron refugios preferidos, es decir refugios que fueron visitados recurrentemente. Para observar esto se decidió graficar la acumulada de noches pasadas en un refugio para los refugios más utilizados por una tortuga. Esto se realizó solo para los datos provenientes de los i-gotU, ya que se contó con una gran cantidad de días consecutivos de medición (Fig. \ref{fig:refugios_preferidos}). 
\begin{figure}[ht]
    \begin{center}
        \includegraphics[width=1.4\imsize]{Chap3/acumulada_de_noches_en_refmasUsados_espanol.pdf}
        \caption[Acumulada de noches pasadas en los refugios preferidos.]{Acumulada de noches pasadas en los refugios preferidos para distintas tortugas monitoreadas por los i-gotU de enero 2022 a mayo 2022.}
        \label{fig:refugios_preferidos}
       
        \end{center}
\end{figure}
Se observa en la Fig. \ref{fig:refugios_preferidos} que las tortugas tienen un refugio preferido donde pasan la mayor parte de las noches monitoreadas. Sin embargo, también se observó que algunas tortugas tienen varios refugios preferidos, como es el caso de la tortuga T54. En las tortugas T30, T54, T6 y T79, se registraron días donde decidió pasar la noche en otro refugio (línea horizontal) para después volver a su refugio preferido (recta con pendiente positiva). 

Para visualizar estas rutas entre refugios y la preferencia relativa entre ciertos refugios, se realizaron mapas con la librería Folium donde se graficaron los refugios con circulos de tamaño proporcional a la cantidad de noches que la tortuga pasó en el refugio, con conexiones entre refugios ilustrando las rutas tomadas. Es decir si la T54 pasó una noche en el refugio 12 y la noche siguiente en el refugio 3, habrá una línea entre el refugio 12 y el refugio 3 en el mapa. En la Fig. \ref{fig:ruta_refus_T54} se muestra un ejemplo de este tipo de mapa para la tortuga T54.
 
\begin{figure}[ht]
    \begin{center}
        \includegraphics[width=\imsize]{Chap3/refugies_path_for_t54.jpg}
        \caption[Caminos tomados entre refugios para la tortuga T54.]{Caminos tomados entre refugios para la tortuga T54. El tamaño de nodo refugio es proporcional a la cantidad de noches que pasó la tortuga en el mismo. Una conexión entre par de nodos refugios aparece solo si pasó una noche en el refugio de origen y la noche siguiente en el refugio de destino. Al clickear sobre un nodo refugio se puede ver la cantidad de noches que pasó la tortuga en el mismo.}
        \label{fig:ruta_refus_T54}
       
        \end{center}
\end{figure}
Haber encontrado preferencia por ciertos refugios abre distintas preguntas: ¿Qué características comparten estos refugios preferidos? ¿Siguen recurriendo a estos refugios en distintas épocas del año u otros años?  Responder estas preguntas podría ayudarnos a entender el rol de la memoria en esta especie de reptiles y a plantear medidas de conservación de la especie. En futuras campañas de medición se espera poder entender más sobre estos refugios preferidos.
 
 
 
%hasta aca corregido 
\section{Redes bipartitas de refugios}
Se armaron redes bipartitas con nodos refugios y nodos tortugas. En este tipo de redes, los nodos refugios solo están conectados con nodos tortugas y las conexiones indican que esa tortuga uso ese refugio. Partiendo de todos los labels de refugios con las tortugas que pasaron noche en ese refugio, se armaron las redes bipartitas para los datos tomados por los tortugometros y los datos tomados con los i-gotU, Figs. \ref{fig:red_bipartita_refus_campanas} y \ref{fig:red_bipartita_refus_igotu} respectivamente.
 
\begin{figure}[ht]
    \begin{center}
        \includegraphics[width=\imsize]{Chap3/RedBipartita_deRefugios_corregida_campanas_2300.pdf}
        \caption[Red bipartita de refugios para los datos de los tortugometros.]{Red bipartita de refugios para los datos de los tortugometros. Las distancias relativas entre nodos y el grosor del link son dependientes de la cantidad de noches que una  tortuga pasó en el refugio. }
        \label{fig:red_bipartita_refus_campanas}
       
        \end{center}
\end{figure}
 
\begin{figure}[ht]
    \begin{center}
        \includegraphics[width=1.3\imsize]{Chap3/RedBipartita_deRefugios_IGOTO.pdf}
        \caption[Red bipartita de refugios para los datos de los i-gotU.]{Red bipartita de refugios para los datos de los i-gotU. Las distancias relativas entre nodos y el grosor del link son dependientes de la cantidad de noches que una  tortuga pasó en el refugio. }
        \label{fig:red_bipartita_refus_igotu}
       
        \end{center}
\end{figure}
En la Figs. \ref{fig:red_bipartita_refus_campanas} y \ref{fig:red_bipartita_refus_igotu}, se observan refugios compartidos entre dos y tres pares de refugios.  En base a este resultado, se decidió buscar la probabilidad de que dos nodos tortugas estén conectados en la red de encuentros (Figs. \ref{fig:redInteraccion20mincampanas} y \ref{fig:redInteraccion20igotu}). Al proyectar la red bipartita en nodos tortugas obtenemos una redes comparables con las redes de encuentros, en las Figs. \ref{fig:proyeccion_red_campanas} y \ref{fig:proyeccion_red_igotu}. Sin embargo la interpretación de ambas redes es diferente. Mientras la red de encuentros une tortugas que se encontraron durante el día, la red proyectada une tortugas que hayan usado el mismo refugio.
 
 
\begin{figure}[ht]
    \begin{center}
        \includegraphics[width=\imsize]{Chap3/Proyecion_bipartita_ref_solo_tortus.pdf}
        \caption[Proyección  de red bipartita de refugios para datos de los tortugometros en nodos tortugas.]{Proyección  de red bipartita de refugios para datos de los tortugometros (Fig. \ref{fig:red_bipartita_refus_campanas}) en nodos tortugas. Si hay una refugio compartido por un par de nodos tortugas, aparece una conexión entre este par de nodos en la proyección. }
        \label{fig:proyeccion_red_campanas}
       
        \end{center}
\end{figure}
 
\begin{figure}[ht]
    \begin{center}
        \includegraphics[width=\imsize]{Chap3/Proyecion_bipartita_ref_solo_tortus_IGOTU.pdf}
        \caption[Proyección  de red bipartita de refugios para datos de los tortugometros en nodos tortugas.]{Proyección  de red bipartita de refugios para datos de los i-gotU (Fig. \ref{fig:red_bipartita_refus_igotu}) en nodos tortugas. Si hay una refugio compartido por un par de nodos tortugas, aparece una conexión entre este par de nodos en la proyección. }
        \label{fig:proyeccion_red_igotu}
       
        \end{center}
\end{figure}
Para  ver que tan probable es el encuentro entre tortugas, en caso que hayan usado alguna vez el mismo refugio, se decidió tomar las proyecciones de las redes bipartitas como predictores de conexiones en las redes de encuentros. Se calcularon las métricas precisión, accuracy y recall, partiendo de las cantidades verdadero positivo (TP), verdadero negativo (TN), falso positivo (FP) y falso negativo (FN). Donde, por ejemplo, TP se calculó como la cantidad de conexiones existentes en las redes de encuentros que también están en las redes proyectadas. La precisión se calculó como TP/(TP+FP), accuracy como (TP+TN)/(TP+TN+FP+FN) y  recall como TP/(TP+FN). Los resultados se muestran en la tabla \ref{tab:metricas_red_bipartita}.

\begin{table}[ht]
    \centering
    \begin{tabular}{|c|c|c|c|}
       
   \hline
    Metodología  & Precisión & Recall & Accuracy \\ \hline
    Tortugometro & 0.125     & 0.059  & 0.258    \\ \hline
    i-gotU       & 1         & 0.4    & 0.4       \\ \hline
   
    \end{tabular}
    \caption[Tabla con métricas de comparación entre redes de encuentros y proyecciones de redes bipartitas.]{Tabla con métricas de comparación entre redes de encuentros y proyecciones de redes bipartitas. Se tomó la proyeccion de la red bipartita como predictor de conexiones en las red de encuentros.}
    \label{tab:metricas_comparacion_redes}
\end{table}

Se observa en la tabla \ref{tab:metricas_comparacion_redes}, que las métricas obtenidas para los datos provenientes de los tortugometros son considerablemente menores que las métricas obtenidas para los datos provenientes de los i-gotU. Estas diferencias se cree que están asociadas a la poca cantidad de refugios nocturnos medidos para los datos del tortugómetro ya que originalmente las campañas de medición no se planearon para este tipo de análisis. Por otro lado, con los i-gotU se monitorean una menor cantidad de tortugas, produciendo una desviación en la medición.
 
Para determinar si los resultados de la tabla \ref{tab:metricas_comparacion_redes} son estadísticamente significativos, se realizaron operaciones de \textit{double edge swap} sobre las redes bipartitas de refugios y se compararon los valores obtenidos sobre estas nuevas redes generadas, después de las proyecciones. Se realizó un código en Python que elige dos conexiones al azar en la red bipartita y las intercambia si es que no existe ya estas conexiones. Es decir, si T10 uso el refugio 54 y T11 el refugio 32, se intercambian los enlaces en caso que T10 no tenga una conexión con el refugio 32 y tampoco la T11 con el 54 \cite{github}, este procedimiento se itera de manera de generar una red aleatoria manteniendo la distribución de grado constante (un equivalente en cierto sentido a mantener la misma cantidad de mediciones pero tomando uso de refugios al azar). Partiendo de 1000 redes generadas a partir de 1000 cambios aleatorios de conexiones (1000 \textit{double edge swap}), se obtuvieron las métricas de precisión, recall y accuracy para cada red generada. Sobre estos valores se calculó la cantidad de veces donde las métricas halladas por usos aleatorios de refugios fueron mayores que los valores obtenidos  de la tabla \ref{tab:metricas_comparacion_redes}. Los resultados porcentajes donde se obtubieron metricas mayores para cada metodologia se muestran en la tabla \ref{tab:metricas_comparacion_redes_aleatorias}.
\begin{table}[ht]
    \centering
    \begin{tabular}{|c|c|c|c|}
       
   \hline
    Metodología  & \% Precisión mayor  &  \% Recall mayor & \% Accuracy mayor \\ \hline
    Tortugometro & 60    & 50  & 50    \\ \hline
    i-gotU       & 0        & 0    & 0      \\ \hline
   
    \end{tabular}
    \caption[Tabla con comparación de métricas obtenidas en redes bipartitas con usos aleatorios de refugios respecto a las métricas medidas.]{Tabla con comparación de métricas obtenidas por redes bipartitas con usos aleatorios de refugios respecto de las métricas medidas. Se tomaron las proyecciones de la redes bipartitas generadas aleatoriamente como predictor de conexiones en las redes de encuentros y se calcularon las proporciones donde estas métricas son mayores a las obtenidas por la tabla \ref{tab:metricas_comparacion_redes}.}
    \label{tab:metricas_comparacion_redes_aleatorias}
\end{table}

Se observa en la tabla \ref{tab:metricas_comparacion_redes_aleatorias}, que las métricas obtenidas para los datos provenientes de los i-gotU son siempre mayor o iguales a lo esperado por uso aleatorio de refugios, mientras que las métricas obtenidas para los datos provenientes de los tortugometros no lo son (es igual de probable obtener metricas mayores o menores mediante el uso aleatorio de refugios nocturos). Esto se debe a que la cantidad de refugios nocturnos medidos para los datos del tortugómetro es considerablemente menor que la cantidad de refugios nocturnos medidos para los datos de los i-gotU. Por otro lado, con los i-gotU se monitorea una menor cantidad de tortugas, haciendo posible la existencia de un desvio de medición.
 
Una posible comparación entre las proyecciones de las redes bipartitas con las redes de encuentros está dada por la topología de las redes.  En la siguiente sección se analiza la topología de las redes de encuentros y se compara con la topología de las redes proyectadas en nodos tortugas, comparando con métricas a partir del uso aleatorio de los refugios.
\section{Comparación de topología de redes de encuentros y redes proyectadas}
Se calcularon las métricas modularidad, densidad de la red, coeficiente de agrupamiento y centralidad de grado medio en nodos tortugas machos y hembras para las redes de encuentros (Figs. \ref{fig:redInteraccion20mincampanas} y \ref{fig:redInteraccion20igotu}) y para las redes proyectadas (Figs. \ref{fig:proyeccion_red_campanas} y \ref{fig:proyeccion_red_igotu}). En la tabla \ref{tab:metricas_topologia_redes}   se muestran los valores obtenidos para las distintas métricas, para los datos provenientes de las dos metodologías.
 
 
 
\begin{table}[ht]
    \centering
    \begin{tabular}{|c|c|c|c|c|}
    \hline
    Métricas          & E. Tortu   & P. Tortu      & E. i-gotU   & P. i-gotU    \\ \hline
    Modularidad       & 0.5         & 0.6           & 0.1        & 0            \\ \hline
    C. agrupamiento & 0.28        & 0.16          & 0           & 0            \\ \hline
    Densidad          & 0.22        & 0.09          & 0.50         & 0.13          \\ \hline
    C.G.M. machos     & $0.2\pm0.1$ & $0.08\pm0.08$ & $0.5\pm0.2$ & 0            \\ \hline
    C.G.M. hembras    & $0.2\pm0.1$ & $0.10\pm0.07$ & 0.5         & $0.2\pm0.1 $ \\ \hline
    \end{tabular}
    \caption[Tabla con métricas asociadas a las tipologías de las redes de encuentros y las redes proyectadas.]{Tabla con métricas asociadas a las topologías de las redes de encuentros y las redes proyectadas. E. y P. se refiere a redes de encuentros y redes proyectadas respectivamente, para cada metodología de medición. C. agrupamiento se refiere a coeficiente de agrupamiento y C.G.M. se refiere a centralidad de grado medio.}
    \label{tab:metricas_topologia_redes}
\end{table}
Podría esperarse, que la centralidad de grado medio de los machos sea mayor que el de las hembras en base de observaciones de campo, especialmente en época de apariamiento (Nov-Dic), pero se encontró en la tabla \ref{tab:metricas_topologia_redes} que no presentan diferencias significativas para los datos provenientes de ambas metodologías \cite{Erika}. Se encontraró menor densidad de red en la red proyectada de tortugas que en la red de encuentros, para ambas metodologías. Esto está relacionado a la poca cantidad de mediciones de refugios compartidos. En el caso de los tortugómetros, el filtro para la selección de refugios es muy estricto, generando una poca cantidad de refugios respecto de la cantidad de días de medición. Por otro lado, en el caso de los i-gotU, la cantidad de refugios medidos es alta pero la mayoría de estos fueron medidos en meses donde la tortuga baja la actividad a causa de las temperaturas del ambiente y decide quedarse la mayor parte de las noches en los refugios preferidos encontrados (Fig. \ref{fig:refugios_preferidos}).
 
Con respecto al coeficiente de agrupamiento, se observa que para los datos provenientes de los tortugómetros, el coeficiente de agrupamiento es distinto de cero. Para el caso de los datos provenientes de los i-gotU, el coeficiente de agrupamiento es cero. Esto se debe a que la red de encuentros y la red proyectada no presentan triángulos entre cualquier triplete de nodos, debido seguramente a la poca cantidad de individuos monitoreados con este método. 

Se encontraron modularidades mayores en los datos provenientes de los tortugómetros en comparación con los datos provenientes de los i-gotU para las redes de encuentros y las redes proyectadas. Esto se debe a la existencia de dos claras comunidades en las redes provenientes de los datos del tortugometro (Figs. \ref{fig:redInteraccion20mincampanas} y \ref{fig:proyeccion_red_campanas}) y a la ausencia de comunidades en las redes de los datos de los i-gotU (Figs. \ref{fig:redInteraccion20igotu} y \ref{fig:proyeccion_red_igotu}).
 
Sobre las redes bipartitas se generaron 1000 redes aleatorias  realizando 1000 operaciones del tipo \textit{double edge swap}. Se compararon las métricas obtenidas en las redes proyectadas de la tabla \ref{tab:metricas_topologia_redes} con las métricas obtenidas en las proyecciones de las redes aleatorias. En la Fig. \ref{fig:distribucion_coef_agrupa}, se muestra un ejemplo de la distribución del coeficiente de agrupamiento en las redes aleatorias generadas partiendo de la red bipartita de refugios con datos de los tortugometros (Fig. \ref{fig:red_bipartita_refus_campanas}). Se observa que no presenta diferencias significativas entre los valores hallados con el valor medido en la proyección de la red original. Se observó el mismo tipo de comportamiento para todas las métricas calculadas en la tabla \ref{tab:metricas_topologia_redes}. Esto quiere decir que no habría diferencias con una red aleatoria.
\begin{figure}[ht]
    \begin{center}
        \includegraphics[width=\imsize]{Chap3/coef_clustering_distribucion.pdf}
        \caption[Distribución del coeficiente de agrupamiento en proyecciones de redes aleatorias.]{Distribución del coeficiente de agrupamiento calculado de las proyecciones en nodos tortugas de mil redes generadas aleatoriamente a partir de la red bipartita con datos de los tortugometros. En rojo está el valor medido del coeficiente de agrupamiento para la proyección en nodos tortugas (Fig. \ref{fig:proyeccion_red_campanas}).}
        \label{fig:distribucion_coef_agrupa}
       
        \end{center}
\end{figure}
 
\section{Proyecciones de redes bipartitas en nodos refugios}
Se realizaron las proyecciones de las redes bipartitas (Figs. \ref{fig:red_bipartita_refus_campanas} y \ref{fig:red_bipartita_refus_igotu}) en nodos refugios, Figs. \ref{fig:proyeccion_red_campanas_refus} y \ref{fig:proyeccion_red_igotu_refus}.
 
 
\begin{figure}[ht]
    \begin{center}
        \includegraphics[width=\imsize]{Chap3/Proyecion_bipartita_ref_solo_ref.pdf}
        \caption[Proyección  de red bipartita de refugios para datos de los tortugometros en nodos refugios.]{Proyección  de red bipartita de refugios para datos de los tortugometros (Fig. \ref{fig:red_bipartita_refus_campanas}). Si hay una refugio compartido por un par de nodos tortugas, aparece una conexión entre este par de nodos en la proyección. }
        \label{fig:proyeccion_red_campanas_refus}
       
        \end{center}
\end{figure}
 
\begin{figure}[ht]
    \begin{center}
        \includegraphics[width=\imsize]{Chap3/Proyecion_bipartita_ref_solo_ref_iGOTU.pdf}
        \caption[Proyección  de red bipartita de refugios para datos de los i-gotU en nodos refugios.]{Proyección  de red bipartita de refugios para datos de los i-gotU (Fig. \ref{fig:red_bipartita_refus_igotu}) sobre los nodos refugios. Si hay una refugio compartido por un par de nodos tortugas, aparece una conexión entre el refugio común con los respectivos refugios de ambas tortugas en la proyección. }
        \label{fig:proyeccion_red_igotu_refus}
       
        \end{center}
\end{figure}
En las Figs. \ref{fig:proyeccion_red_campanas_refus} y \ref{fig:proyeccion_red_igotu_refus}, se observan pequeños clusters donde hay nodos completamente conectados (asociados a los refugios que visitó una tortuga) con algunos nodos que conectan distintos clusters (asociados a algún refugio compartido). Ejemplo de este nodo conector es el refugio 4 (Fig. \ref{fig:proyeccion_red_igotu_refus}), que fue utilizado por la tortuga T30 y T6 en distintas noches (Fig. \ref{fig:red_bipartita_refus_igotu}).
 
Una pregunta subyacente de las proyecciones calculadas es si existe alguna relación entre los links de la red y las distancias geograficas entre los nodos refugios. Para responder esta pregunta se graficaron los refugios en el mapa junto con las conexiones dadas por los links en las proyecciones. Un ejemplo de este mapa para los datos de los i-gotU se muestra en la Fig. \ref{fig:mapa_con_conexiones_igotu}. Se observa que parte de los links se encuentran entre refugios vecinos, pero también hay unos pocos links entre refugios que se encuentran a distancias considerables respecto de refugios vecinos. A falta de una relación más rigurosa entre las distancias y los links, se realizó un \textit{mantel test} \cite{MantelTest} entre las matrices de adyacencia de las redes proyectadas en nodos refugios (Figs. \ref{fig:proyeccion_red_campanas_refus} y \ref{fig:proyeccion_red_igotu_refus}) con matrices de distancias entre refugios. La matriz de adyacencia esta definida como: la matriz booleana que representa las conexiones entre pares de nodos refugios. En el lugar i,j de la matriz de distancias se encuentra la distancia entre el refugio  de la posición i y el refugio j (en metros) de la matriz de adyacencia.
 
El mantel test calcula el coeficiente de correlación de Pearson entre estas dos matrices, luego realiza permutaciones aleatorias de la matriz de distancias y vuelve a calcular el coeficiente de correlación de Pearson. El p-valor es la proporción de permutaciones que dan un coeficiente de correlación de Pearson mayor o igual al coeficiente de correlación de Pearson de la matriz de distancias original. Bajo la hipótesis de correlación nula en las dos matrices, las permutaciones aleatorias deberían ser igualmente probables de producir valores mayores o menores del coeficiente de correlación calculado.
 
 
Los mantel tests realizados con 10000 permutaciones aleatorias para los datos de los tortugometros y los i-gotU dan un p-valor de 0.0051 y 0.0001 respectivamente. Esto indica que existe una correlación significativa entre las distancias entre refugios y los enlaces en las redes proyectadas en nodos refugios (Figs. \ref{fig:proyeccion_red_campanas_refus} y \ref{fig:proyeccion_red_igotu_refus}). Es decir que la tortuga que visita un refugio, tiene una probabilidad mayor de visitar refugios cercanos a este, como se observa en la Fig. \ref{fig:mapa_con_conexiones_igotu}. Esto también se evidencia observando la trayectoria de visita de los refugios, siendo la regla que la tortuga posee noches consecutivas en refugios cercanos entre si (Fig. \ref{fig:ruta_refus_T54}).
 
\begin{figure}[ht]
    \begin{center}
        \includegraphics[width=\imsize]{Chap3/Mapa_refus_con_coneciones_igotu.jpg}
        \caption[Proyección en nodos refugios con conexiones en el mapa.]{Proyección de red bipartita (Fig. \ref{fig:proyeccion_red_igotu_refus}) en nodos refugios con conexiones en el mapa para los datos de i-gotU. }
        \label{fig:mapa_con_conexiones_igotu}
       
        \end{center}
\end{figure}
 
 
 
 
 
 


\chapter{Conclusiones y trabajo a futuro}
 
Se diseñó un algoritmo de reconstrucción de trayectorias, el mismo puede ser usado en cualquier especie indicando la velocidad máxima de la misma. Se encontró que la velocidad máxima de las tortugas es de 14 m/min. Fueron identificadas las zonas más visitadas por la población de tortugas estudiadas (Fig.~\ref{fig:grillaInt}), lo que podría ser útil para diseñar políticas de conservación. Los mapas de recurrencias podrán ser actualizados con nuevas mediciones y al devolver mapas en un formato html interactivo son de fácil uso para un trabajo interdisciplinario, en particular, durante las campañas.
 
Se obtuvieron redes  de interacción de individuos (Figs. \ref{fig:redInteraccion20mincampanas} y \ref{fig:redInteraccion20igotu}) y la cantidad de encuentros promedio para machos y hembras en los distintos meses de medición (Figs.~\ref{fig:encuentros_hora_medida_tortugometro} y \ref{fig:encuentros_hora_medida_igotu}). Se observa un pico para ambos sexos en noviembre-diciembre para los datos del tortugometro, donde el 85\% de los encuentros registrados fueron hembra-macho, éste puede estar asociado a la búsqueda de pareja en la época de apareamiento. En el mes de enero las hembras presentan una mayor cantidad de encuentros promedio por hora medida que los machos (para datos de i-gotU y tortugometro), esta observación se estima que puede estar relacionada a la búsqueda de un lugar para sus huevos. De todas maneras, solo se contaron con 8 tortugómetros y 6 i-gotU  para cada día midiendo en simultáneo y es probable que parte de los encuentros no hayan sido registrados. En un futuro se analizará la red de interacciones aumentando el número de pares de individuos monitoreados.
 
 
Se definieron dos criterios para determinar el refugio donde pasó la noche la tortuga, uno para cada metodología de medición. Reconociendo las limitaciones de estos criterios, se decidió para las próximas campañas de medición aumentar las franja horaria de medición para los datos de i-gotU, de esta manera garantizar la ubicación de la tortuga en el refugio nocturno. Para el caso del tortugómetro se decidió añadir una etiqueta si la tortuga se encuentra en su refugio nocturno a la hora de recuperar el dispositivo en ese día de medición.
 
Con los refugios ya identificados, se definió una métrica para determinar la distribución espacial de los refugios en tortugas machos y hembras. Se observó que no presentan diferencias significativas entre machos y hembras, de todas maneras, el criterio utilizado para determinar un refugio para datos del tortugometro filtra muchos días de medición donde se le quitó el tortugómetro previo a las 20 horas. Por otro lado, los datos de i-gotU (donde tenemos una gran cantidad de refugios registrados) están tomados en los meses de enero-mayo, donde se espera una menor actividad de las tortugas machos (pasada época de apareamiento)\cite{Erika}.
 
Se identificaron en los refugios monitoreados por los i-gotU, donde tenemos días consecutivos de medición, la existencia de refugios preferidos (Fig. \ref{fig:refugios_preferidos}) donde la tortuga pasa la mayoría de las noches. En 4 de 6 tortugas monitoreados con metodología i-gotU, se encontró que realizan caminatas desde el refugio preferido hasta otro refugio nocturno y luego vuelven al refugio preferido. Un ejemplo de este último es la tortuga T54, en la red que manifiesta los caminos tomados (Fig. \ref{fig:ruta_refus_T54}), se observa que la tortuga  toma varias caminatas a otros refugios nocturnos y luego vuelve a alguno de sus dos refugios preferidos. Caracterizar estos refugios preferidos puede ser útil para diseñar políticas de conservación y puede además contribuir a la caracterización del movimiento, tanto para modelos matemáticos como simulaciones numéricas.
 
Partiendo de los refugios asociados a cada tortuga se armaron redes bipartitas de refugios y tortugas (Figs. \ref{fig:red_bipartita_refus_campanas} y \ref{fig:red_bipartita_refus_igotu}). Utilizando las proyecciones de los grafos bipartitos como predictor de enlaces en la red de encuentros, se encontró que para los datos de los tortugometros las métricas recall, precisión y accuracy no son mayores de lo esperado por uso de refugios aleatorio (tabla \ref{tab:metricas_comparacion_redes_aleatorias}). Para los datos de i-gotU, se observa que las métricas son mayores a las esperadas por uso de refugios aleatorios, pero se están monitoreando solo 6 tortugas y únicamente se encontraron 2 refugios compartidos entre las tortugas. Por poca cantidad de datos, no se puede aún afirmar si las proyecciones en la red bipartita nos dan una predicción de la red de encuentros.
 
Además, se compararon las topologías de las redes proyectadas en nodos tortugas con las redes de encuentros (tabla \ref{tab:metricas_topologia_redes}). Se observa que las redes bipartitas de refugios y tortugas tienen una topología similar a la red de encuentros, pero con una menor cantidad de enlaces y tampoco presentan diferencias significativas respecto el uso aleatorio de refugios.  En un futuro se analizará este razonamiento con una mayor cantidad de tortugas donde se disponga la misma cantidad de datos de refugios que de días de medición (condición que actualmente no se cumple para los datos del tortugometro).
 
Sobre las redes bipartitas de nodos tortugas y refugios, se realizaron proyecciones en nodos refugios (Figs. \ref{fig:red_bipartita_refus_campanas} y \ref{fig:red_bipartita_refus_igotu}) y se encontró que las matrices de adyacencia están fuertemente correlacionadas con la matrices de distancias. Se obtuvo un p-valor de 0.0051 y 0.0001, para los datos de los tortugometros y los i-gotU respectivamente, al realizar mantel tests con 10000 permutaciones aleatorias. Es decir que la tortuga que visita un refugio, tiene una probabilidad mayor de visitar refugios cercanos a este, como se observa en la Fig. \ref{fig:mapa_con_conexiones_igotu}.
 
 
 
 
 
 
 
 
 
 
%A partir de los resultados de este estudio, se decidió realizar en la próxima campaña un mayor esfuerzo de seguimiento de los individuos que participaron de los encuentros  detectados. De esta forma se busca entender si la no repetición de encuentros entre tortugas se debe a la poca cantidad de muestras o a una característica de la especie. Este resultado sería muy novedoso dado que está relacionado con la capacidad de memoria de las tortugas, un aspecto muy poco estudiado hasta el momento.
 
 
 
 



%\appendix
%\include{apend1}

\begin{biblio}
\bibliography{mibib}
\end{biblio}


\begin{postliminary}

% \begin{seccion}{Publicaciones asociadas}
%   \begin{enumerate}
%   \item Mi primer aviso en la revista \textbf{ABC}, 1996
%   \item Mi segunda publicaci\'{o}n en la revista \textbf{ABC}, 1997
%   \end{enumerate}
% \end{seccion}

\begin{seccion}{Agradecimientos}
A todos los que se lo merecen, por merecerlo!
\end{seccion}

\end{postliminary}

\end{document}

