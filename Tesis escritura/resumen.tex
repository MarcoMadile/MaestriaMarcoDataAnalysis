\begin{resumen}%
    Sobre la tortuga \textit{Chelonoidis chilensis} se sabe muy poco, es una especie en estado vulnerable  y actualmente, en la zona de estudio, se esta introduciendo ganado; haciendo muy importante el estudio de  sus refugios, su area de movimiento y las relaciones entre tortugas dentro la comunidad. La población de estudio está en la distribución más al sur. Además, por ser reptil se consideran solitarios y se sabe muy poco sobre su red de interacción social.
    En este trabajo, se estudió el movimiento de 27 individuos de \textit{Chelonoidis chilensis} usando dos técnicas de monitoreo: una unidad de navegación  autónoma con GPS y un datalogger comercial. Se implementó un método de filtrado de trayectorias y se construyó una grilla de zonas de interés para las tortugas, utilizando las trayectorias filtradas e interpoladas. Se implementaron dos criterios para identificar los refugios nocturnos de las tortugas. Sobre estos refugios se calculó la distancia media entre refugios al centro de masa (de estos refugios) para machos y hembras y no se encontraron diferencias significativas. Se armaron redes bipartitas de nodos tortugas y refugios y se compararon las proyecciones en nodos tortugas con redes de interacción, armadas a través de los encuentros medidos entre pares de tortugas. Se utilizó la proyección en nodos tortugas de la red bipartita como predictor de enlaces en la red de encuentros y se encontró que las predicciones no son estadísticamente significativas. Se calcularon métricas sobre la topología de la red proyectada y no se encontraron diferencias respecto al uso aleatorio de refugios. Se observaron la existencia de refugios preferidos y se concluyó que la tortuga pasa la noche en refugios cercanos respecto a otros refugios medidos en la zona de medición.
\end{resumen}

\begin{abstract}%
english abstract
\end{abstract}


%%% Local Variables: 
%%% mode: latex
%%% TeX-master: "template"
%%% End: 
