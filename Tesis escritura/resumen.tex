\begin{resumen}%
    Se estudió el movimiento de 27 individuos de \textit{Chelonoidis chilensis} usando dos técnicas de monitoreo: una unidad de navegación  autónoma con GPS y un datalogger comercial. Se implementó un método de filtrado de trayectorias y se construyó una grilla de zonas de interés para las tortugas, utilizando las trayectorias filtradas e interpoladas. Se implementaron dos criterios para identificar los refugios nocturnos de las tortugas. Sobre estos refugios se calculó la distancia media entre refugios al centro de masa (de estos refugios) para machos y hembras y no se encontraron diferencias significativas. Se armaron redes bipartitas de nodos tortugas y refugios y se compararon las proyecciones en nodos tortugas con redes de interacción, armadas a travez de los encuentros medidos entre pares de tortugas. 
\end{resumen}

\begin{abstract}%
english abstract
\end{abstract}


%%% Local Variables: 
%%% mode: latex
%%% TeX-master: "template"
%%% End: 
