\chapter{Uso de refugios}
Para varias especies los refugios son fundamentales para la protección de predadores y  condiciones climaticas (especialmente para animales de sangre fría, como las tortugas). En especies relativamente solitarias, los individuos pasan un tiempo considerable solos en los refugios y tienen pocos encuentros directos fuera de la epoca de apareamiento. %citar foto con encuentros por epoca? solo para campanas? 
Ejemplos de estas especies incluyen a los mapaches, zorros rojos, orangutanes y algunas especies de abejas, avispas y murcielagos. Para estas poblaciones de animales salvajes, monitorear y entender estos refugios puede ayudar a establecer patrones sociales de los individuos. 

En distintos campos cercanos a la zona de medición (San Antonio Oeste, provincia de Río Negro) estan introduciendo ganado al abitad de las tortugas, es necesario entender si presentan una amenaza para la integridad de los refugios y  patron de movimiento  de las tortugas sobre los mismos junto con las caracteristicas de los refugios preferidos para tomar medidas de conservación de la especie. 

\section{Redes bipartitas de refugios}
Se armaron redes bipartitas con nodos refugios y nodos tortugas. Los nodos refugios solo estan conectados con nodos tortugas. Para determinar el refugio donde paso la noche la tortuga se tomó el ultimo punto de  de la tortuga en un día de medición y se pidió la condicion de que haya sido tomado despues de las 19 horas. A este punto nuevo se le asigna un label de refugio y un enlace con la tortuga que paso la noche en ese refugio. A medida que se añade otro refugio primero se verifica que presente una distancia mayor a 20 metros con todos los otros refugios labeleados, en caso que la distancia es menor a 20 metros a por ejemplo el refugio 1, se le asigna el label del refugio 1 a este nuevo punto. Partiendo de todos los labels de refugios y tortugas se armaron las redes bipartias para los datos tomados por los tortugometros y los datos tomados con los i-gotU.

\begin{itemize}
    \item Criterio para identificar refugio, redes bipartitas de refus
    \item Distribucion espacial de los refugios con la definicion que usepackage
    \item Paths de refugios y acumulada de noches en refugios (material adicional gif de refugies paths)
    \item proyeccion en solo refugios, mapa con proyeccion y metricas asociadas 
    \item Mantel test entre matriz de distancias y matriz adyacencia de refus 
\end{itemize}
\section{Comparacion red encuentros con bipartita de refugio}
Capaz esta section podria ser otro capitulo y aca poner las metricas alladas.
\begin{itemize}
    \item Usar una como predictor de conexiones 
    \item Comparar metricas obtenidas 
    \item Double edge swapping 
\end{itemize}