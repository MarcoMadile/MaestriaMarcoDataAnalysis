%%%%%%%%%%%%%%%%%%%%%%%%%%%%%%%%%%%%%%%%%%%%%%%%%%%%%%%%%%%%%%%%%%%%%%%%%%%%%%%%
% \documentclass[12pt,papel,twoside]{ibtesis}
\documentclass[12pt,screen,twoside,pagebackref]{ibtesis}
% \documentclass[12pt,papel,singlespace,oneside]{ibtesis}
% \documentclass[12pt,papel,preprint,singlespace,oneside]{ibtesis}


%%%%%%%%%%%%%%%%%%%%% Paquetes extra %%%%%%%%%%%%%%%%%%%%%%%%%%%%%%%%%%%%%%%%%%%
% Por conveniencia: aqu\'{\i} puede cargar todos los paquetes y definir los %comandos que necesite
\usepackage{ibextra}
\usepackage{amsmath}
\usepackage{algorithm}
\makeatletter
\renewcommand{\ALG@name}{Algoritmo}%Cambio el titulo del entorno algoritmo por %el idioma.
\makeatother
\usepackage[noend]{algpseudocode}
\usepackage[dvipsnames]{xcolor}
%\usepackage[chapter]{algorithm}

%%%%%%%%%%%%%%%%%%%%%%%%%%%%%%%%%%%%%%%%%%%%%%%%%%%%%%%%%%%%%%%%%%%%%%%%%%%%%%%%
%%%%%%%%%%%%%%%%%%%%% Informacion sobre la tesis %%%%%%%%%%%%%%%%%%%%%%%%%%%%%%%
%\title{Titulo de la tesis}
%\author{}
%\director{}
%\codirector{}
%\carrera{Tesis Carrera de Maestr\'{\i}a en F\'isica/Ingenier'{\ia}
%\grado{Maestrando}
%\laboratorio{F\'{i}sica Estad\'{i}stica e Interdisciplinaria -- Centro At\'{o}mico Bariloche}
%\jurado{}
%\palabrasclave{}
%\keywords{}
% Si queremos poner la fecha manualmente:
%\date{Diciembre de 2099}

%%%%%%%%%%%%%%%%%%%%%%%%%%%%%%%%%%%%%%%%%%%%%%%%%%%%%%%%%%%%%%%%%%%%%%%%%%%%%%%%
%\titlepagefalse % Si no quiere compilar la portada descomente esta linea
%\includeonly{apendices} % Compilar s\'{o}lo estos archivos 
%\graphicspath{{includes/}} % Lugar donde encontrar las figuras generales (se puede poner uno en cada cap{\'{\i}}tulo)
%%%%%%%%%%%%%%%%%%%%%%%%%%%%%%%%%%%%%%%%%%%%%%%%%%%%%%%%%%%%%%%%%%%%%%%%%%%%%%%%
%\renewcommand{\i}{\italic{i}}
\begin{document}
%\maketitle
\section*{Bitácora - Maestría IB Marco Madile Hjelt- }

Este template sirve a su vez como plantilla del formato Oficial de tesis de maestría IB, cuando se descomenta lo que está comentado y se completa en los campos correspondientes tal como está indicado.\\ 

\subsection{Reunión 12 de Agosto de 2022}\\

KARINA: En esta primera reunión, Marco nos mostró la distribución de links obtenida para los encuentros en espacio y tiempo de las tortugas que veniamos monitoreando hasta ahora con tortugometro.\\
Creamos un proyecto en Github MaestriaIBMarco, donde agregué el libro de modelos para datos de telemetría de Hooten et. al. y un artículo que encontré sobre redes de interacción social de tortugas y también los nuevos datos de los 6 IgotoGPS que estuvieron monitoreando desde Enero hasta ahora.\\
Quedamos en que Marco comenzaría a hacer:\\
1) Con Igraph o networkx la matriz de adyacencia a partir de los datos. Los nodos son las tortugas y los links indican quien se encontró con quien.\\ 
2) Aumentar el grosor de links indicando cuantas veces se repitió el encuentro.\\
3) Hacer el mismo análisis con los nuevos datos de Igoto (15 días) para comparar resultados con los datos de tortugometro que están en una ventana temporal más chica (4 días).\\
4) Importante: Marco, fijate que hay una hoja que dice radiotransmisores10y30, son datos de gps medidos a mano, luego de rastrear a la tortuga con antena Yagui. Entonces seria importante verificar que esas coordenadas coinciden (dentro de cierto error) con las reportadas por en Igotu. Veo que no están ordenadas las fechas...lo que puede complicar la comparación....pero bueno hagamoslo esta vez para convencernos de que coinciden (por ejemplo yo dibujaria los puntos de tx de otro color arriba de los otros para ver si coinciden o no). Los datos de T10b.csv y T30b.csv deberían ser los mismos que los T10.csv y T30.csv de la otra carpeta.\\ 

5) Armar la red con las madrigueras nocturnas (esto lo acabo de agregar je).\\

Cosas curiosas que encontramos: la cantidad de conexiones a partir de 7 (?) sube cuando relajo el tiempo en el que cuento los eventos de encuentros espaciotemporales.\\
Adjunto un .bib pero ojo! que se que tiene muchos errores, igual sirve a modo de guía bibliográfica, no lo corrijan porque ya lo tengo corregido por ahi, luego se los paso.\\

Marco hasta reunion del 25 de agosto de 2020: \\
1) En carpeta primeras redes, arme redes de interacción de tortugas con distintas condiciones de encuentro.\\
2) agregué al codigo que el tamaño del edge sea dependiente del número de veces que  repitieron el encuentro.\\
3) agregué una función que me devuelva un dicionario de nombres de tortugas a sexo a partir de los datos guardados en csv.\\ 
4) con mismo dicionario pinte los nodos de rosa, azul y gris  para las tortugas machos, hembras y desconocidas respectivamente.\\ 
5) arme codigo para leer datos del Igoto y calcular los encuentros con este nuevo formato de archivos (excel de distintas columnas con formato variable sobre el mismo archivo).\\
6) sobre esta tarde (24/8) voy a armar las redes con estos nuevos ajustes y datos. 

\subsection*{Reunión 25 de Agosto}

KARINA, ideas para la reunión: 
0) Identificar sexo de tortugas T6, T54, T128, T184.\\
1) Identificar los refugios identificando lat, lon donde paso la noche la tortuga y dibujarlos sobre el mapa\\
2) Clasificar los "tipos de encuentros entre tortugas", si son encuentros largos (i. e., persecucion macho-
hembra o compartieron refugio por ejemplo) o si son encuentros cortos.\\ 
3) Encuentro trabajo "Inferring social structure and its drivers from refuge
use in the desert tortoise, a relatively solitary species" Sah et. al. 2016 (referencia subida a la carpeta bibliografia)
, algunas conclusiones que nos pueden servir como ideas:\\
"We show that seasonal variation has a strong impact on
tortoise burrow switching behavior.\\
"popular refuges can be used to iden-
tify core habitat areas."\\
"sudden changes in the
refuge switching behavior of individuals can be used as an
early warning signal of disturbances that may ultimately
affect population fitness."\\



4) Dos posibles papers?: \\
    a) Redes de interacción entre tortugas, estudio de las escalas temporales (igoto vs. tortugometro), 
    implicancia desde lo social y relevancia en la transmisión de enfermedades.\\
    b) Modelos de movimiento inspirados en los datos (distribucion de tamaño de paso y tiempo de espera), 
    ajustes de los modelos a los datos. Posible adaptación/extensión de los modelos de Tesis Laila para el monito? 
    por ejemplo caminatas correlacionadas (CRW). Para luego hacer el ajuste a los datos.\\

5) Pensar que tortugas conviene monitorear en la próxima campaña en funcion de estos resultados. Por ejemplo
monitorear con igoto las mismas tortugas que se han monitoreado para las redes?\\

6) Normalizar el ancho de los links (con la suma de encuentros de todos los dias de todas las tortugas (?)) para poder ver diferencia en la cantidad de encuentros durante 20min, 5 dias y 15dias
para que viendo las tres redes nos demos cuenta a simple vista de la cantidad de encuentros.

7) hacer funcion general para calcular simultaneidad en el set de datos, con felxibilidad para cambiar la ventana 
de tiempo (minutos a dias por ejemplo) y para diferenciar dia de noche.

%%%%%%%%%%%%%%%%%%%%%%%%%%%%%%%%%%%%%%%%%%%%%%%%%%%%%%%%%%%%%%%%%
%lo siguiente va debajo de begindocument y antes de comenzar la introduccion de la tesis.
% Dentro del environment 'preliminary' va:
% la dedicatoria, resumen, abstract, indices

%\begin{preliminary}
  
    % Escriba su dedicatoria
    %\dedicatoria{}

    %%% \'{I}ndices %%%%

   %\begin{abreviaturas}
                                %Abreviaturas
    %                                 GPU: Unidad de Procesamiento Gr\'afico %(Graphics Processing Unit)\\
    %\end{abreviaturas}
    
    %\tableofcontents                %\'{I}ndice
    
    %\listoffigures                  %Figuras
    
    %\listoftables                   %Tablas
    
    %\begin{resumen}%
    Sobre la tortuga \textit{Chelonoidis chilensis} se sabe muy poco, es una especie en estado vulnerable  y actualmente, en la zona de estudio, se esta introduciendo ganado; haciendo muy importante el estudio de  sus refugios, su area de movimiento y las relaciones entre tortugas dentro la comunidad. La población de estudio está en la distribución más al sur. Además, por ser reptil se consideran solitarios y se sabe muy poco sobre su red de interacción social.
    En este trabajo, se estudió el movimiento de 27 individuos de \textit{Chelonoidis chilensis} usando dos técnicas de monitoreo: una unidad de navegación  autónoma con GPS y un datalogger comercial. Se implementó un método de filtrado de trayectorias y se construyó una grilla de zonas de interés para las tortugas, utilizando las trayectorias filtradas e interpoladas. Se implementaron dos criterios para identificar los refugios nocturnos de las tortugas. Sobre estos refugios se calculó la distancia media entre refugios al centro de masa (de estos refugios) para machos y hembras y no se encontraron diferencias significativas. Se armaron redes bipartitas de nodos tortugas y refugios y se compararon las proyecciones en nodos tortugas con redes de interacción, armadas a través de los encuentros medidos entre pares de tortugas. Se utilizó la proyección en nodos tortugas de la red bipartita como predictor de enlaces en la red de encuentros y se encontró que las predicciones no son estadísticamente significativas. Se calcularon métricas sobre la topología de la red proyectada y no se encontraron diferencias respecto al uso aleatorio de refugios. Se observaron la existencia de refugios preferidos y se concluyó que la tortuga pasa la noche en refugios cercanos respecto a otros refugios medidos en la zona de medición.
\end{resumen}

\begin{abstract}%
english abstract
\end{abstract}


%%% Local Variables: 
%%% mode: latex
%%% TeX-master: "template"
%%% End: 
 //si quiero incluir un archivo resumen.tex que esté en este mismo directorio

%\end{preliminary}

% Podemos usar cualquiera de los dos comandos: \input o \include para incluir el texto
%\chapter{Introducción}
\chapterquote{Avanza rapido el tema de las tortugas}{D. H. Zanette}

\section{ C. \textit{chilencis} }
\label{S:opciones-que-acepta}

La mayoría de las especies animales son capaces de realizar complejos patrones de movimientos que generalmente dependen del ambiente, factores intrínsecos de los individuos y las interacciones entre ellos (\cite{morales2005adaptive}, \cite{morales2010building} y \cite{nathan2008emerging}). La complejidad de estos movimientos están manifestados en sus trayectorias. Para el caso de las tortugas estas trayectorias dependen fuertemente de la vegetación en la zona de estudio y la época del año (caso que busquen reproducirse, depositar sus huevos, etc.).


Nuestra especie de interés es la tortuga $Chelonoidis$ $chilensis$. Se distribuye desde el Gran Chaco hasta el norte de la Patagonia, como se muestra en la Fig.~\ref{fig:distribuciondeEspecie} (\cite{chebez2008se}). Esta especie está incluida en el Appendix  de la \textit{Convention on International Trade in Endangered Species of Wild Fauna and Flora (CITES)} y fue categorizada como \textit{vulnerable} a nivel nacional \cite{prado2012categorizacion} e internacional por la \textit{International Union for Conservation of Nature (IUCN)}.
Los principales factores que llevaron a esta situación son la reducción, modificación y destrucción de su hábitat, debido a la expansión de la frontera agropecuaria, y su comercialización, siendo la especie nativa de reptiles más ilegalmente traficada en el mercado de mascotas en Argentina (\cite{prado2012categorizacion}). Además, la amenaza a esta especie se ve aumentada con la introducción de especies depredadoras exóticas como el Jabalí (\textit{Sus scrofa}) (\cite{kubisch2014chelonoidis}). En este trabajo estudiaremos una población de tortugas en en el límite sur de su distribución geográfica, a 20 km al norte de San Antonio Oeste, provincia de Río Negro.  \\
    
Las tortugas son animales  herbívoros que se alimentan con tallos y frutos de cactus (\textit{Opuntia sulphurea, Cereus aethiops, Perocactus tuberosus}), gramíneas (\textit{Chloris castilloniana, Trhichloris crinita}), herbáceas (\textit{Alternanthera pugens, Sphaeralcea miniata, S. mendocina, Portulaca grandiflora}) y vainas de leguminosas (\cite{zacarias2016biologia}).
    
    
    
\begin{figure}[h]
    \begin{center}
        \includegraphics[width=\imsize]{figs/Chap1/Chelonoidis_chilensis_geographic_range.png}
        \caption{Distribución geográfica de la especie de tortuga \textit{Chelonoidis chilensis} \label{fig:distribuciondeEspecie}.}
        
        \end{center}
\end{figure}
Esta especie presenta un dimorfismo sexual cuando son adultos. Los machos son notablemente más chicos que las hembras. El período de actividad en la distribución más sur de la especie es el más corto, ya que bruman (parecido a hibernar) por aproximadamente cinco meses. Sus períodos de actividad comienzan en el mes de septiembre y, desde noviembre a diciembre, es cuando el apareamiento es mayormente observado. Entre enero y marzo es cuando las hembras pasan una gran parte del tiempo buscando un lugar adecuado para enterrar sus huevos \cite{Erika}. Todavía falta mucho por aprender acerca de la biología de la población de C. \textit{chilencis} presente en Argentina.
 
Motivados por la falta de información, el objetivo de este estudio es caracterizar el movimiento de las tortugas en una de las poblaciones en el límite sur de su distribución geográfica. Aprender acerca del movimiento de los individuos es fundamental para entender su rol ecológico en el ecosistema y para diseñar políticas de conservación de la especie y su hábitat.


\section{Metodologias}
%%% Local Variables: 
%%% mode: latex
%%% TeX-master: "template"
%%% End: 

%\chapter{Redes de interacción entre tortugas}
\graphicspath{{figs/}}

\chapterquote{In retrospect, Euler's unintended message is very simple: Graphs or networks have properties, hidden in their construction, that limit or enhance our ability to do things with them.}{Albert-László Barabási, 1982}

\label{Redes de interacción entre tortugas}
\section{Trayectorias }

Primero se muestran las trayectorias obtenidas para un día de medición (Fig.~\ref{fig:trayeSinFiltr}), como por ejemplo 1/12/2020. Para éstas se realizó un programa en el lenguaje Python utilizando la librería Folium, permitiendo añadir puntos de GPS al mapa \cite{github}.
 
\begin{figure}[ht]
    \begin{center}
       
   
    \includegraphics[width=\imsize]{Chap2/Traye1_12_sinF.png}
\end{center}
    \caption[Trayectorias un dia de medición, sin filtrrar.]{Trayectorias del 1/12/2020, cada color representa una tortuga diferente. Ambas metodologías fueron implementadas, algunos puntos tomados con el tortugómetro escapan a la trayectoria esperada.}
    \label{fig:trayeSinFiltr}
\end{figure}
Se observa en la Fig.~\ref{fig:trayeSinFiltr}, que algunos puntos tomados por el tortugómetro se desvían de la trayectoria esperada para una tortuga (recorren distancias del orden de los kilómetros en menos de 10 minutos). Se estima que estas desviaciones se producen por dos motivos: en primer lugar, en los primeros minutos de medición, el GPS comienza a conectarse a satélites hasta tener la precisión máxima, haciendo que  los primeros puntos tengan una mayor desviación; en segundo lugar, se observó de manera aleatoria la desviación de algún punto respecto de la trayectoria típica.
 
 
 
 
Para corregir estas desviaciones, se implementó un método basado en la velocidad máxima que pueden alcanzar los individuos. El mismo está detallado en el repositorio de GitHub, archivo \textit{CriterioParaSacarData.py} \cite{github}. Para obtener la velocidad máxima, se calculó la distribución de velocidades de la Fig.~\ref{fig:distribuciondeVel}.

 
\begin{figure}[ht]
\begin{center}
       
   
    \includegraphics[width=\imsize]{Chap2/Velocidades2.jpeg}
    \caption[Distribución de velocidades.]{Histograma de velocidades en m/min. Las  velocidades obtenidas mayores a 15 m/min están órdenes de magnitud por encima.}
    \label{fig:distribuciondeVel}
\end{center}
\end{figure}
Se observó en la distribución de velocidades de la Fig.~\ref{fig:distribuciondeVel}, que las tortugas llegan a una velocidad máxima de aproximadamente 15m/min, de manera que se adoptó el criterio de filtrar los tramos de trayectoria en los que la velocidad supera ese valor máximo. Filtrando los puntos de la Fig.~\ref{fig:trayeSinFiltr}, tomando velocidad máxima 15 m/min, se obtuvo  el mapa de la Fig.~\ref{fig:trayeConFiltr}.
 
 
 

\begin{figure}[ht]
    \begin{center}
       
   
    \includegraphics[width=\imsize]{Chap2/Traye1_12_conF.png}
\end{center}
    \caption[Trayectorias un dia de medición, despues del filtrado.]{Trayectorias del 1/12/2020 luego del filtrado, cada color representa una tortuga diferente.}
    \label{fig:trayeConFiltr}
\end{figure}

\section{Zonas de interes}
Partiendo de las trayectorias filtradas, se realizó  una grilla identificando las zonas de recurrencia en la Fig.~\ref{fig:grilla1}. La misma es de gran interés para la investigación y preservación de la especie, en caso que se pudieran identificar los factores o características de las zonas más recurridas, se podrían sugerir políticas de manejo para minimizar los daños sobre las tortugas. Esto es especialmente importante dado que ahora se está introduciendo ganado en la zona con el consiguiente deterioro del hábitat natural de las tortugas. La grilla fue programada en Python y asigna la cantidad de posiciones medidas por el GPS en cada celda \cite{github}.
 
 
\begin{figure}[ht]
    \begin{center}
    \includegraphics[width=\imsize]{Chap2/GrillaSintCNoche.png}
    \end{center}
    \caption[Mapa de zonas de recurrencia.]{Mapa de recurrencias  interactivo con las trayectorias filtradas. Al hacer clic en cualquier celda de la grilla un cartel dice cuantas mediciones fueron tomadas. El tamaño de celda es de 10m$^2$.}
    \label{fig:grilla1}
\end{figure}
 
Se puede observar en la Fig.~\ref{fig:grilla1} un punto que se destaca mucho más que el resto (arriba a la izquierda) teniendo el máximo de mediciones en esa casilla. Esto se debe a que una pareja de tortugas pasó la noche con el tortugómetro puesto en medio de un arbusto de difícil acceso. Para obtener una mejor idea de las zonas de interés diurnas se realizó otra grilla usando sólo datos del tortugómetro registrados en el día (entre 7am y 9pm) y  realizando una interpolación lineal de 1 punto por minuto por cada par de puntos consecutivos (Fig.~\ref{fig:grillaInt}, \cite{github}). Esta interpolación da una aproximación de las casillas por donde tuvo que pasar la tortuga y añade un peso cuando la tortuga se quedó dentro de la misma casilla por una mayor cantidad de tiempo (mediciones consecutivas).
 
\begin{figure}[ht]
    \begin{center}
        
    
    \includegraphics[width=\imsize]{Chap2/GrillaCintSNoche.png}
\end{center}
    \caption[Mapa con zona de recurrencia para trayectorias diurnas.]{Mapa de recurrencias  interactivo con las trayectorias diurnas (7am-9pm) filtradas e interpoladas linealmente. Al hacer clic en cualquier celda de la grilla un cartel dice cuantas mediciones fueron tomadas. El tamaño de celda es de 10m$^2$.}
    \label{fig:grillaInt}
\end{figure}
 
Comparando las Figs. \ref{fig:grilla1} y \ref{fig:grilla1}, se observa en la  Fig.~\ref{fig:grillaInt} un mayor contraste de las otras celdas respecto al que se encuentra arriba a la izquierda. Esto se debe a la extracción de los puntos nocturnos.  Si se puede obtener las características de las zonas de interés, se podrían sugerir políticas de manejo para minimizar los daños sobre las tortugas. Esto es especialmente importante dado que ahora se está introduciendo ganado en la zona con el consiguiente deterioro del hábitat natural de las tortugas.

\section{Red de encuentros}
Partiendo de las trayectorias filtradas, se decidió buscar el solapamiento de las trayectorias, para identificar los encuentros. Para ello, se implementó un codigo en Python, que partiendo de cualquier punto de su trayectoria busca si hay otro punto de otra tortuga que se encuentre a una distancia menor a 20 metros y a una distancia temporal menor a 20 minutos. Cuando se cumple esta condición se van guardando los pares de puntos junto con la hora y el nombre de ambas tortugas.

En las Figs. \ref{fig:encuentros_hora_medida_tortugometro} y \ref{fig:encuentros_hora_medida_igotu}, están la cantidad de encuentros calculados por hora medida por los tortugómetros y por los i-gotU en función de los meses de medición. Los encuentros del tipo macho-hembra fueron normalizados por la cantidad promedio de horas medidas de ambos sexos para cada mes, en cambió para la cantidad de encuentros macho-macho y hembra-hembra se normalizo utilizando la cantidad de horas medidas para cada sexo en cada mes.  
\begin{figure}[ht]
    \begin{center}
       
   
    \includegraphics[width=\imsize]{Chap2/encuentros_por_hora_tortugometro.pdf}
\end{center}
    \caption[Encuentros por hora medida tomando los datos del tortugometro.]{Encuentros sobre cantidad de horas medidas para cada sexo en función de los meses de medición utilizando el tortugometro.Los distintos colores identifican el tipo de encuentro.}
    \label{fig:encuentros_hora_medida_tortugometro}
\end{figure}

\begin{figure}[ht]
    \begin{center}
       
   
    \includegraphics[width=\imsize]{Chap2/encuentros_por_hora_igotu.pdf}
\end{center}
    \caption[Encuentros por hora medida tomando los datos de los i-gotU.]{Encuentros sobre cantidad de horas medidas para cada sexo en función de los meses de medición utilizando los i-gotU. Los distintos colores identifican el tipo de encuentro.}
    \label{fig:encuentros_hora_medida_igotu}
\end{figure}
En la Fig. \ref{fig:encuentros_hora_medida_tortugometro}, se observa que el máximo de encuentros del tipo macho-hembra por hora medida esta en los meses noviembre-diciembre, esto coincide con la epoca de apariamiento. Estos meses estan juntos ya que las mediciones en esos meses fueron tomadas a finales de noviembre y principios de diciembre. Para el mes de enero solo se registraron encuentros del tipo hembra-hembra en ambas figuras ( \ref{fig:encuentros_hora_medida_tortugometro} y \ref{fig:encuentros_hora_medida_igotu}), esto puede deberse a que las hembras están buscando un lugar acorde para depositar sus huevos, haciendo el encuentro hembra-hembra más probable. Para los datos de los i-gotU (\ref{fig:encuentros_hora_medida_igotu}) tambien se registraron encuentros en los meses de febrero, marzo y abril, pero en menor cantidad que en los meses anteriores, asociamos esta diferencia a la disminución de actividad en las tortugas.


Utilizando los encuentros calculados, se armaron dos redes de interacción  en la librería NetworkX \cite{networkx}, una para los datos obtenidos utilizando  tortugometro y otra para los datos provenientes de i-gotU (Figs. \ref{fig:redInteraccion20mincampanas} y \ref{fig:redInteraccion20igotu}). Las conexiones entre nodos tortugas tienen peso linealmente dependiente de la cantidad de encuentros entre ellas, esto se observa en el grosor del link entre dos tortugas y las distancias relativas entre nodos.


\begin{figure}[ht]
    \begin{center}
       
   
    \includegraphics[width=\imsize]{Chap2/red_interaccion_20min_campanas.pdf}
\end{center}
    \caption[Red de encuentros entre tortugas  con datos tomados por el tortugometro.]{Red de encuentros entre tortugas para datos provenientes de las metodología  tortugometro. La condición de encuentro esta dada por una distancia espacial menor a 20 metros y a una distancia temporal menor a 20 minutos.}
    \label{fig:redInteraccion20mincampanas}
\end{figure}



\begin{figure}[ht]
    \begin{center}
       
   
    \includegraphics[width=\imsize]{Chap2/red_interaccion_20min_IGOTO.pdf}
\end{center}
    \caption[Red de encuentros entre tortugas utilizando i-gotU.]{Red de encuentros entre tortugas para datos provenientes de las metodologías i-gotU. La condición de encuentro esta dada por una distancia espacial menor a 20 metros y a una distancia temporal menor a 20 minutos.}
    \label{fig:redInteraccion20igotu}
\end{figure}


\begin{Huge}
Idea : conectar con criterio de encuentros mostrar grafico de encuentros segun la epoca del año y pasar a las dos redes de interacción que tenemos 
\end{Huge}

%%% Local Variables: 
%%% mode: latex
%%% TeX-master: "template"
%%% End: 


%\appendix
%\include{apend1}


\begin{biblio}
\nocite{*}
\bibliography{referencestortoises}
\end{biblio}

%\begin{postliminary}

    %\begin{seccion}{Agradecimientos}
    
    %\end{seccion}

%\end{postliminary}

\end{document}